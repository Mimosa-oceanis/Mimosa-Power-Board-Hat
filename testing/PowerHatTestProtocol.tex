% !TEX TS-program = pdflatex
% !TEX encoding = UTF-8 Unicode

% This is a simple template for a LaTeX document using the "article" class.
% See "book", "report", "letter" for other types of document.

\documentclass[10pt]{article} % use larger type; default would be 10pt

\usepackage[utf8]{inputenc} % set input encoding (not needed with XeLaTeX)

%%% Examples of Article customizations
% These packages are optional, depending whether you want the features they provide.
% See the LaTeX Companion or other references for full information.

%%% PAGE DIMENSIONS
\usepackage{geometry} % to change the page dimensions
\geometry{a4paper} % or letterpaper (US) or a5paper or....
% \geometry{margin=2in} % for example, change the margins to 2 inches all round
% \geometry{landscape} % set up the page for landscape
%   read geometry.pdf for detailed page layout information

\usepackage{graphicx} % support the \includegraphics command and options

% \usepackage[parfill]{parskip} % Activate to begin paragraphs with an empty line rather than an indent

%%% PACKAGES
\usepackage{booktabs} % for much better looking tables
\usepackage{array} % for better arrays (eg matrices) in maths
\usepackage{paralist} % very flexible & customisable lists (eg. enumerate/itemize, etc.)
\usepackage{verbatim} % adds environment for commenting out blocks of text & for better verbatim
\usepackage{subfig} % make it possible to include more than one captioned figure/table in a single float
% These packages are all incorporated in the memoir class to one degree or another...

%%% HEADERS & FOOTERS
\usepackage{fancyhdr} % This should be set AFTER setting up the page geometry
\pagestyle{fancy} % options: empty , plain , fancy
\renewcommand{\headrulewidth}{0pt} % customise the layout...
\lhead{}\chead{}\rhead{}
\lfoot{}\cfoot{\thepage}\rfoot{}

%%% SECTION TITLE APPEARANCE
\usepackage{sectsty}
\allsectionsfont{\sffamily\mdseries\upshape} % (See the fntguide.pdf for font help)
% (This matches ConTeXt defaults)

%%% ToC (table of contents) APPEARANCE
\usepackage[nottoc,notlof,notlot]{tocbibind} % Put the bibliography in the ToC
\usepackage[titles,subfigure]{tocloft} % Alter the style of the Table of Contents
\renewcommand{\cftsecfont}{\rmfamily\mdseries\upshape}
\renewcommand{\cftsecpagefont}{\rmfamily\mdseries\upshape} % No bold!
\usepackage{varwidth}
\usepackage[graphicx]{realboxes}
\usepackage{rotating}%
%%% END Article customizations

%%% The "real" document content comes below...

\title{Mimosa Power Hat Test Protocol}
\author{Jérôme Béguin}
%\date{} % Activate to display a given date or no date (if empty),
         % otherwise the current date is printed 

\begin{document}
\newcolumntype{M}{>{\begin{varwidth}{12cm}}l<{\end{varwidth}}} %M is for Maximal column
\maketitle

\section{Power}

First step before connecting the board to a Raspberry Pi, ensure the voltage regulator \emph{U4} works correctly. 
\begin{itemize}
\item Connect 12V power supply to \emph{J4 PWR IN} connector. 
\item Measure 12V on Pin 1 \& 2 of \emph{U4}. 
\item Measure 5V on Pin 10 \& 11 of \emph{U4}
\item Measure 5V on Pin 2 \& 4 of 40 Pins connector
\item Measure with an oscilloscope the stability of 5V on Pin 2 \& 4 of 40 Pins connector
\end{itemize}

If these firsts are passed it's possible to plug the board on to a raspberry pi. 
Then test the other voltages.

\begin{itemize}
\item Measure 3.3V on Pin 1 of 40 pins connector
\item Measure 3.3V on Pin 2 of \emph{J5} connector
\end{itemize}

\section{Real Time Clock}
Real time clock chip is the \emph{U5} chip on the board. It is connected to raspberry pi via $I^2C$ to Raspberry Pi. \emph{SDA} on pin 3 and \emph{SCL} on pin 5 of 40 pins connector.

\textbf{NOTE}
On the version 0 of the Power hat PCB the coin cell enclosure is too small to accept 3V coin cell battery. It will be corrected in future version of the board.

\begin{itemize}
\item Find $I^2C$ device on Raspberry pi using
\begin{verbatim}
i2cdetect -y 1
\end{verbatim}
and find the adresse 0xD2 or 0xD3
\item it then should be possible to setup and use the RTC with some Raspberry pi library. 
\end{itemize}

\section{PWM Fan control}
The connector \emph{M1 CPU FAN} is supposed to be connected to a 5V PWM computer fan.
\begin{itemize}
\item Using Raspberry pi, create a PWM signal on Pin 32 (GPIO 12) of 40 pins connector.
\item Measure PWM signal on Pin 4 of \emph{M1 CPU FAN} connector with 5V amplitude. 
\item If a Fan is connected on \emph{M1 CPU FAN} connector a 3V feedback signal of rotation speed of the fan should be sent back on Pin 22 (GPIO 24) of 40 pin connector.
\end{itemize}

\section{NMEA0183 Interface}
The connector \emph{J1} is the \emph{NMEA0183} interface. It is connected to the raspberry pi via serial connection on Pin 8 \& 10 (GPIO 14 \& 15) of 40 Pins connector. The \emph{NMEA0183} connector is a \emph{RS422} port with one differential input and one differential output.
\begin{itemize}
\item Connect an \emph{RS422} \emph{NMEA0183} divise like a GPS on \emph{J1} connector
\item Receive \emph{NMEA0183} phrases on serial port of raspberry pi.
\end{itemize}

\section{NMEA2000 CAN bus}
The connector \emph{J2} is a CAN interface. It is supposed to be connected on a CAN bus as a device and should be able to communicate with the other device on the bus. The CAN is translated into SPI and connected to raspberry pi through 40 pins connector via Pins 19, 21 \& 23 (GPIO 10, 9 \& 11) for MOSI, MISO \& CLK. Chip select (CS) is on Pin 26 (GPIO8). 
\begin{itemize}
\item Connect \emph{J2} connector to a CAN bus. 
\item receive CAN messages on SPI on raspberry pi.
\end{itemize}


\end{document}
